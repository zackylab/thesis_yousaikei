\begin{abstract}
   Nodeは,ElixirがErlang言語から継承するライブラリ関数である.Broadway は,Elixir チームによってデータパイプラインを作成および管理できるツールである,軽量なプロセスモデルと耐障害性を兼ね備えている.我々は,Nodeのマルチノード耐障害性とBroadwayのデータ処理機能を画像処理パイプラインとして活用することに着目する.本研究では画像処理を行うBroadwayパイプラインを構築し,その中でNodeを用いることで,耐障害性を向上させたパイプラインの開発と評価を行うことである.さらに Elixir のスーパーバイザを導入することで耐障害性を持たせた.提案手法を実装してその性能を評価し,NodeとBroadwayにおける画像処理の適用可能性および耐障害技術課題を明らかにする.
\begin{center}
  \textbf{Abstract}
\end{center}
  Node is a library function that Elixir inherits from the Erlang language. Broadway is a tool that allows the Elixir team to create and manage data pipelines, combining a lightweight process model with fault tolerance. We focus on utilizing Node's multi-node fault tolerance and Broadway's data processing capabilities as an image processing pipeline. In this research, we will construct a Broadway pipeline that performs image processing, and use Node within it to develop and evaluate a pipeline with improved fault tolerance. In addition, we made fault tolerance by introducing Elixir supervisor. We implemented the proposed method and evaluated its performance, and clarified the applicability of image processing in Node and Broadway and the challenges of fault tolerance technology.
\end{abstract}





