\begin{abstract}
   Nodeは,ElixirがErlang言語から継承するライブラリ関数である.Broadway は,Elixir チームによってデータパイプラインを作成および管理できるツールである,軽量なプロセスモデルと耐障害性を兼ね備えている.我々は,Nodeのマルチノード耐障害性とBroadwayのデータ処理機能を画像処理パイプラインとして活用することに着目する.提案手法として画像処理を行うBroadwayパイプラインを構築し,その中でSupervisorとNodeを用いることで,耐障害性を向上させたパイプラインの実装を行うことである.さらに提案する Broadway パイプラインを実装してその性能を評価して, NodeとBroadwayにおける画像処理の適用可能性および耐障害技術課題を明らかにする.結果として, 安全性では, 実際にメモリ不足などによる異常終了が発生した場合, 再起動ができた. また, NIF 関数の使用時 abort を発生する場合, 起動元の Erlang VM に波及されなかった. 回復速度では, 実際にメモリ不足などによる異常終了が発生した場合, 高速な復帰を実現できた. しかし, エラーが発生したときに以前の状態に安全にロールバックしてデータの不整合を回避することはできなかった. 
\begin{center}
  \textbf{Abstract}
\end{center}
  Node is a library function that Elixir inherits from the Erlang language. Broadway is a tool that allows the Elixir team to create and manage data pipelines, combining a lightweight process model with fault tolerance. We focus on utilizing Node's multi-node fault tolerance and Broadway's data processing capabilities as an image processing pipeline. The proposed method is to construct a Broadway pipeline that performs image processing, and to implement a pipeline with improved fault tolerance by using Supervisors and Nodes. Furthermore, we implement the proposed Broadway pipeline and evaluate its performance, clarifying the applicability of image processing in Node and Broadway and issues in fault tolerance technology. As a result, in terms of safety, if an abnormal termination due to lack of memory or the like actually occurred, it was possible to restart.Also, if an abort occurred when using the NIF function, it did not affect the Erlang VM that started it. In terms of recovery speed, if an abnormal termination due to memory shortage etc. actually occurred, a fast recovery could be achieved. However, when an error occurs, it is necessary to safely roll back to the previous state to avoid data inconsistency.
\end{abstract}





